\chapter{Multigroup Reactor Methodology}
\index{Multigroup Reactor Methodology@\emph{Multigroup Reactor Methodology}}

\section{Introduction}
\index{Introduction@\emph{Introduction}}
Nothing to see here.

\section{Multigroup Cross Section Generatrion}
\index{Multigroup Cross Section Generatrion@\emph{Multigroup Cross Section Generatrion}}
When seeking to parameterize nuclear power reactors as a function of initial conditions, 
the set of possible indpendent parameters quickly becomes large. In addition to geometric 
design considerations, the fuel characteristics of the reactor must also be accounted for.

In the one-energy-group reactor model (R1G), the initial loading was paramterized based
on neutron production rates, neutron destruction rates, and transmutation matricies per
nuclide \cite{Scopatz2009d}.  All of these metrics are a function of flunece.  (Under 
constant irradiation, fluence is a clear surrogate for time.)

A $G$-energy-group reactor model (RMG) seeks to re-paramterize the one-group formulation 
in terms of energy.  While the multi-group formulation will remain on a per nuclide basis, 
the neutron reaction rates do not have have a meaningful per unit energy expression that 
is independent of the flux.  Moreover changing reaction rates also invalidate the 
transmutation matricies.  

Therefore, the RMG, in a level of sophistication above the R1G, must be able to calculate
the multigroup flux.  To do so requires multigroup microscopic neutron cross-sections.  
Using the cross-sections as independent reactor parameters in a multigroup sense is 
effectively equivelent to removing the flux from the reaction rates in the one-group case.
This is seen in Equation \ref{reaction_rate_calc}
\begin{equation}
\label{reaction_rate_calc}
R = \sigma \cdot \phi \cdot 10^{-24}
\end{equation}
where $R$ [hz] is the reaction rate, $\sigma$ [barns] is the one-group cross section, and
$\phi$ [n/cm\superscript{2}/s] is the energy-integrated flux.

\subsection{Notation}
\index{Notation@\emph{Notation}}
The group constants $\sigma_{itg}$ are themselves parameterized by nuclide, time, and 
incident neutron energy.  Nuclides are indexed by $i$, times are indexed by $t$, and 
energy is indexed by $g$ with the lower indices representing higher energy groups.

\begin{table}[htbp]
\begin{center}
\caption{Neutron Reaction Types}
\label{reaction_type_table}
\begin{tabular}{|l||c|c|}
\hline
\textbf{Tally}                              & \textbf{Symbol} & \textbf{MT} \\
\hline
Total                                       & $t$             & 1  \\
Scattering                                  & $s$             & 2 + 4 \\
Elastic Scattering                          & $e$             & 2 \\
Inelastic Scattering                        & $i$             & 4 = sum(51, 91) \\
$n$\superscript{th}-state Inelastic Scatter & $i\{n\}$        & 50 + $n$ \\
(n, 2n)                                     & $2n$            & 16 \\
(n, 3n)                                     & $3n$            & 17 \\
Fission                                     & $f$             & 18 = 19 + 20 + 21 +38 \\
First-chance Fission                        & $f19$           & 19 \\
Second-chance Fission                       & $f20$           & 20 \\
Third-chance Fission                        & $f21$           & 21 \\
Fourth-chance Fission                       & $f38$           & 38 \\
Absorption                                  & $a$             & 27 = 18 + sum(102, 107) \\
Neutron Capture                             & $\gamma$        & 102 \\
Proton                                      & $p$             & 103 \\
Deuterium                                   & $d$             & 104 \\
Tritium                                     & \nuc{H}{3}      & 105 \\
Helium-3                                    & \nuc{He}{3}     & 106 \\
Alpha                                       & $\alpha$        & 107 \\
Metastable Neutron Capture                  & $\gamma*$       & $\dagger$ \\
Metastable (n, 2n*)                         & $2n*$           & $\dagger$ \\
\hline
\end{tabular}
\end{center}
\end{table}


To fully describe a reactor, several neutron reation types are required.  The type 
also augments the group constant notation by being a comma-separated preposition to 
the index.  For example, the total cross section would be represented by the symbol 
$\sigma_{t,itg}$. Table \ref{reaction_type_table} displays the reactions used in this 
study, their symbolic abbreviations, and the cooresponding MT number coming from ENDF 
specification \cite{MFMT}.  Entries whose MT number is given as a $\dagger$ indicates 
that the generation of these tallies was performed in a special way (see below).
Additionally, the group-to-group scattering cross section is denoted by the symbol
$\sigma_{s,itgh}$ where $g$ denotes the incident neutron energy (as before) and $h$
gives the exiting neutron energy.

\subsection{Parameterization of Initial Conditions}
\index{Parameterization of Initial Conditions@\emph{Parameterization of Initial Conditions}}
The multigroup reactor model requires a library of pre-computed group constants which it uses
to calculate run-time cross section values for the core.  This library must therefore satisfy 
two conditions.  The first is that it contain group constant information for all independent, 
mutable parameters of interest.  The second is that the values of these variables must span 
their cooresponding range of interest.  

Changes in these parameters would then elicit changes in the cross section values. Differences
in the group contants would thus be picked up by the RMG.  For example, take the case of 
neutron self-shielding in a material with a strong resonance aborption peak.  As the number 
denisty of the absorober increases, the group constant in the spectrum around the peak may 
plument because the flux bottoms out.  In a relatively dilute medium, the group constant 
and the flux would increase because fewer neutrons are destroyed in this regime.  That the 
cross section library capture such effects is the primary advantage of a multigroup model
over the traditional one-group method.

\begin{table}[htbp]
\begin{center}
\caption{CHAR Parameters that Define a Perturbation}
\label{char_perturbable_variables}
\begin{tabular}{|l|c|c|}
\hline
\textbf{Parameter}            & \textbf{Symbol}      & \textbf{Units} \\
\hline
Fuel Density                  & $\rho_{\mbox{fuel}}$ & g/cm\superscript{3}  \\
Cladding Density              & $\rho_{\mbox{clad}}$ & g/cm\superscript{3}  \\
Coolant Density               & $\rho_{\mbox{cool}}$ & g/cm\superscript{3}  \\
Fuel Cell Radius              & $r_{\mbox{fuel}}$    & cm \\
Void Cell Radius              & $r_{\mbox{void}}$    & cm \\
Cladding Cell Radius          & $r_{\mbox{clad}}$    & cm \\
Unit Cell Pitch               & $\ell$               & cm \\
Number of Burn Regions        & $b_r$                &  \\
Fuel Specific Power           & $p_s$                & MW/kgIHM \\
Initial Nuclide Mass Fraction & $T_{i0}$             & kg\subscript{i}/kgIHM \\
Burn Times                    & $t$                  & days \\
\hline
\end{tabular}
\end{center}
\end{table}

The code which produces the cross section library is known as CHAR (CITEME).  Char currently
has that ability to adjust a reactor template based on a many inital parameters.  These fall
conceptually into three categories: geometric properties, material properties, and time.
Table \ref{char_perturbable_variables} lists the parameters that define a \emph{perturbation}
in char.


\section{Multigroup Reactor Model}
\index{Multigroup Reactor Model@\emph{Multigroup Reactor Model}}
We

\section{Benchmark}
\index{Benchmark@\emph{Benchmark}}
Can

\section{Conclusion \& Future Work}
\index{Conclusion \& Future Work@\emph{Conclusion \& Future Work}}
Do it!
