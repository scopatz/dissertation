\chapter{Multigroup Reactor Methodology}
\index{Multigroup Reactor Methodology@\emph{Multigroup Reactor Methodology}}

\section{Introduction}
\index{Introduction@\emph{Introduction}}
Nothing to see here.

\section{Multigroup Cross Section Generatrion}
\index{Multigroup Cross Section Generatrion@\emph{Multigroup Cross Section Generatrion}}
When seeking to parameterize nuclear power reactors as a function of initial conditions, 
the set of possible indpendent parameters quickly becomes large. In addition to geometric 
design considerations, the fuel characteristics of the reactor must also be accounted for.

In the one-energy-group reactor model (R1G), the initial loading was paramterized based
on neutron production rates, neutron destruction rates, and transmutation matricies per
nuclide \cite{Scopatz2009d}.  All of these metrics are a function of flunece.  (Under 
constant irradiation, fluence is a clear surrogate for time.)

A $G$-energy-group reactor model (RMG) seeks to re-paramterize the one-group formulation 
in terms of energy.  While the multi-group formulation will remain on a per nuclide basis, 
the neutron reaction rates do not have have a meaningful per unit energy expression that 
is independent of the flux.  Moreover changing reaction rates also invalidate the 
transmutation matricies.  

Therefore, the RMG, in a level of sophistication above the R1G, must be able to calculate
the multigroup flux.  To do so requires multigroup microscopic neutron cross-sections.  
Using the cross-sections as independent reactor parameters in a multigroup sense is 
effectively equivelent to removing the flux from the reaction rates in the one-group case.
This is seen in Equation \ref{reaction_rate_calc}
\begin{equation}
\label{reaction_rate_calc}
R = \sigma \cdot \phi \cdot 10^{-24}
\end{equation}
where $R$ [hz] is the reaction rate, $\sigma$ [barns] is the one-group cross section, and
$\phi$ [n/cm\superscript{2}/s] is the energy-integrated flux.

\section{Multigroup Reactor Model}
\index{Multigroup Reactor Model@\emph{Multigroup Reactor Model}}
We

\section{Benchmark}
\index{Benchmark@\emph{Benchmark}}
Can

\section{Conclusion \& Future Work}
\index{Conclusion \& Future Work@\emph{Conclusion \& Future Work}}
Do it!
