\chapter{Introduction}
\index{Introduction@\emph{Introduction}}
\label{diss_intro}
The nuclear fuel cycle (NFC) has important similarities and differences to other
methods of producing energy and electricity. 
Moreover, there are many possible fuel cycle tracks that may be implemented in a nuclear power economy.
The ability of nuclear power to utilize its own waste stream affords has distinct advantages.  
Foremost among these is the option to limit the number of deep geologic repositories (DRG) 
that must be built to dispose of waste.   All reactors generate fission products (FP) from 
which further fissioning is not possible.  The vast majority of fuel cycles call for the FP masses 
to be buried.  DGR space is limited and precious, particularly in light of the United States Senate 2009
decision to close the Yucca Mountain Project.  Therefore closing the fuel cycle allows for the 
additional electricity to be produced per hypothetical DGR.  For future conceivable needs, 
it is possible to close the cycle such that only one repository needs to be built.  

Partial recycle scenarios in which only one or two 
elements from waste streams still closes the fuel cycle.  For instance, Mixed-oxide (MOX) 
scenarios generally recycle only the plutonium stream.  
It it therefore important to simulate the nuclear fuel cycle.  This involves the characterization 
of material flows at each stage in terms of mass, isotopics, time, cost, \emph{etc}.  This enables 
the coupling of nuclear fuel cycle component design to the design of system as whole.  
Strongly coupling design considerations requires that the fuel cycle be simulated repeatedly, each 
time perturbing some aspect of the cycle.  Thus it is desirable for
NFC simulations to run quickly.  This in turn requires that component simulation is even faster. 
By capturing only the essential physics in component models, commensurate algorithmic speed 
boosts are obtained.

The work presented herein demonstrates two essential physics models of nuclear power reactors.
This in turn allows the simulation of the nuclear fuel cycle on a previously unobtainable level; over 
100000 distinct fuel cycles where considered.  Due to the quantity of fuel cycle data now 
available, new analysis techniques, borrowing heavily from information theory, were developed 
and applied.

The constituent portions of this document are all in-publication (\S \ref{1g_paper}-\ref{ses_paper})
or are currently under-review (\S \ref{cts_paper}-\ref{mg_paper}).  This obviates the need
for framework discussions (such as a global literature review) as each section individually
contains the relevant information.  The portions presented chronologically and illustrate the 
evolution of fuel cycle needs and analysis.

In \S \ref{1g_paper}, a one-energy group reactor model is demonstrated.  Then \S \ref{ses_paper}
uses this essential physics model to simulate a sampling of once-through and closed fuel cycles
which are perturbations of well known base-case cycles.  Next \S \ref{cts_paper} dramatically
expands the space analyzed by removing the dependence on base-case cycles and stochastically 
modeling the NFC as a whole.  However, the single energy group reactor restricts types of reactors
that may be modeled with adequate fidelity.  Thus \S \ref{mg_paper} presents a multigroup reactor
model which incorporates spectral changes as a function of burnup.  Finally, concluding remarks
are presented.

