\chapter{Introduction}
\index{Introduction@\emph{Introduction}}
\label{diss_intro}

The nuclear fuel cycle (NFC) has important similarities and differences to other
methods of producing energy and electricity.  The most commonly known 
fuel cycle is that of carbon burning.  Carbon polymer chains
are either extracted from the Earth's crust or harvested. They are then chemically separated,
releasing energy.  Waste products (CO\subscript{2}, \emph{et al}) have historically been 
stored in the atmosphere.

The nuclear fuel cycle operates similarly.  Heavy metal resources (U, Th) are removed from the ground
and fissioned in a reactor, releasing energy.  Generally, this energy is converted 
into electricity while the excess process heat is released to the environment.  
After the fuel is burned, it is removed from the reactor and stored as a solid on the surface.
This is known as a the once-through fuel cycle.

While there are advantages and disadvantages to each method, an overall comparison of the two is beyond the scope
of this work.  Rather, various future options for the nuclear fuel cycle alone will be discussed.  
Pursuit of the nuclear cycle is motivated by a global, scientific consensus that the carbon cycle 
is having adverse affects on the planetary climate system \cite{Bates2008}.  
Since the peaceful use of nuclear power 
poses none of these climatological problems, the nuclear energy industry stands poised to be utilized 
as a valid, mature alternative to present carbon-based energy resources.
Even in the absence of trying to replace carbon-based methods, 
to sustain the present proportion of the electricity market that comes from 
nuclear power new plants must be built.

There are many possible fuel cycle strategies that may be implemented in a nuclear power economy.
The ability of nuclear power to recycle its own waste stream affords it distinct advantages.  
Foremost among these is the option to limit the number of deep geologic repositories (DGR) 
that must be built to dispose of waste.   All reactors generate fission products (FP) from 
which further fissioning is not possible.  The vast majority of fuel cycles call for the FP masses 
to be buried.  DGR space is limited and precious, particularly in light of the United States Senate 2009
decision to cease work on the Yucca Mountain Project.  
By closing the fuel cycle it is possible  that only one repository need be built to satisfy 
future conceivable needs.

Repository space is not the sole consideration for NFC designers.  Economic considerations 
are also weighted very heavily.  The total cost of electricity from nuclear power must remain competitive 
with other forms of production.  As all components in the cycle contribute to the overall cost burden, 
different strategies \& technologies used may have disparate levelized electricity costs.

However, increased costs may be deemed acceptable if there is a commensurate value added.
For instance, system designers often seek to improve the implicit resistance a fuel cycle 
has to the proliferation of weapons.  Other considerations include natural resource 
utilization and sustainability, operating capacity, dynamic deployment effects,
embodied energy costs, and political feasibility.  The material balance for a given strategy
affects all of these cycle wide metrics.

Because NFCs allow for recycling, material balance calculations may require a higher degree of
algorithmic sophistication than other forms of electricity production.  
Recycle scenarios in which only one or two 
elements from waste streams are re-burned may partially close the fuel cycle.  
Mixed-oxide (MOX) strategies, such as those pursued in France, 
generally recycle only the plutonium stream.  
To compare alternative recycle strategies it therefore important to simulate the nuclear fuel cycle.  
This involves the characterization 
of material flows at each stage in terms of mass, isotopics composition, and time.  This enables 
the coupling of nuclear fuel cycle component design to the design and evaluation of the system as whole.  

Moreover, perturbing a single parameter in the NFC may have global reach over the entire cycle.
For example, in the case of the once-through fuel cycle, altering the initial fresh-fuel 
\nuc{U}{235} enrichment given to a standard light-water reactor changes how much natural 
uranium must be mined earlier in the cycle.  On the back end, enrichment changes also affect 
how much energy may be extracted from the fuel form and how the waste may be safely disposed.

Due to the high degree of interconnectedness between components 
even in the simplest cycles, the need for a dynamic 
fuel cycle simulator and analysis framework arises.

Hence strongly coupling design considerations requires that the fuel cycle be simulated repeatedly, each 
time perturbing some aspect of the cycle.  
Moreover, decision analysis methods often require many iterations over the design option space
to function in a statistically meaningful way.
Thus it is desirable for
NFC simulations to run quickly.  This in turn requires that component simulation is even faster. 
By capturing only the essential physics in component models, commensurate algorithmic speed 
boosts are obtained.

Essential physics models are fuel cycle component algorithms which remain physically valid in the 
locality on which they are defined.  At a minimum, their inputs are \emph{perturbable} 
within an acceptable range and their outputs respond accordingly.  However, such 
models do not seek to compute extraneous parameters that are not of direct importance
to the system at hand.  For example, the flux in the fuel region is of a reactor is 
pertinent to the discharge material composition.  However, the neutron current in the 
shielding is not. 
Essential physics models seek reasonable simplifications of more detailed computational 
simmulations.

The work presented herein will demonstrate two essential physics models of nuclear power reactors.
This allows the simulation of the nuclear fuel cycle on a previously unobtainable level.
For statistical fidelity, over 100000 distinct fuel cycle realizations should be analyzed.  
However, traditional analysis techniques may no longer be sufficient due to the quantity of fuel cycle data now 
available. For example, attempting to perform parametric fits of a large multi-dimensional 
space are often computationally prohibitive and error-prone.
Alternative analysis techniques, such as those from information theory, may prove useful
if essential physics simulators are performant.

The remainder of this document proceeds as follows.  
A fluence-based reactor model will be developed.  This method seeks to characterize nuclear power
reactors on properties of the material initially loaded into the core.  The nuclides
themselves will be parametrized in terms of the time-evolution of the neutron production 
and destruction rates, their burnup, and their transmutation vectors.  Since no discretization
will be done in energy or solid angle, the model will focus exclusively on computing 
material flows.  Thus this method will have the fewest number of algorithmic steps while 
remaining perturbable. 

Once a quickly executing reactor model is demonstrated, it will be applied within a fuel
cycle context.  This will show the validity of such a model as a tool for system designers.
Analyzing several strategies, such as a standard once-through fuel cycle, a recyclable uranium
cycle, and a fast burner reactor cycle, will test the dynamic properties of the fluence-based
model.  

Having proved the reactor model inside of various cycles, the NFC framework will be ready to 
perform at scale.  Picking the fast burner cycle from above, as it has the highest degree of complexity, 
many (stochastically chosen) realizations may be performed.  With this hitherto unseen amount of data, 
fundamentally new types of analysis will be needed to parse through the information.  Entropy-based measures
will be considered as a surrogate for traditional linear sensitivity studies.

After considering the abilities and limitations of this multi-scale model, a refinement to 
the original fluence-based reactor model will be proposed.  In many cores, the flux spectrum
evolves along with the composition.  Such effects would not be captured by the previously 
proposed reactor model.  This in turn limits the fuel cycle schema that may be analyzed.
The integration of a this multi-energy group model which 
accounts for internal spectral shifts into an NFC analysis framework follows
analogously to the work above.

In \S \ref{1g_paper}, the one-energy group reactor model is demonstrated and \S \ref{ses_paper}
uses this essential physics model to simulate a sampling fuel cycles which are perturbations of 
well known base-case cycles.  Next, \S \ref{cts_paper} dramatically
expands the space analyzed by stochastically 
modeling the NFC as a whole. \S \ref{mg_paper} presents a multigroup reactor
model which incorporates spectral changes as a function of burnup.  Finally, concluding remarks
are presented.

