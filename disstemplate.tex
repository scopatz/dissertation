%%%%%%%%%%%%%%%%%%%%%%%%%%%%%%%%%%%%%%%%%%%%%%%%%%%%%%%%%%%%%%%%%%%%%%
%%  disstemplate.tex, to be compiled with latex.                    %%
%%  08 April 2002 Version 5                                         %%
%%%%%%%%%%%%%%%%%%%%%%%%%%%%%%%%%%%%%%%%%%%%%%%%%%%%%%%%%%%%%%%%%%%%%%
%%                                                                  %%
%%  Writing a Doctoral Dissertation with LaTeX at                   %%
%%  the University of Texas at Austin                               %%
%%                                                                  %%
%%  (Modify this ``template'' for your own dissertation.)           %%
%%                                                                  %%
%%%%%%%%%%%%%%%%%%%%%%%%%%%%%%%%%%%%%%%%%%%%%%%%%%%%%%%%%%%%%%%%%%%%%%


\documentclass[12pt]{report}    % The documentclass must be ``report''.

\usepackage{utdiss2}            % Dissertation package style file.


%%%%%%%%%%%%%%%%%%%%%%%%%%%%%%%%%%%%%%%%%%%%%%%%%%%%%%%%%%%%%%%%%%%%%%
% Optional packages used for this sample dissertation. If you don't  %
% need a capability in your dissertation, feel free to comment out   %
% the package usage command.                                         %
%%%%%%%%%%%%%%%%%%%%%%%%%%%%%%%%%%%%%%%%%%%%%%%%%%%%%%%%%%%%%%%%%%%%%%

% Some packages to write mathematics.
\usepackage{amsmath,amsthm,amsfonts,amscd} 
\usepackage{eucal}      % Euler fonts
\usepackage{verbatim}   % Allows quoting source with commands.
\usepackage{makeidx}    % Package to make an index.
\usepackage{psfig}      % Allows inclusion of eps files.
\usepackage{epsfig}     % Allows inclusion of eps files.
\usepackage{citesort}   % 
\usepackage{url}		% Allows nice typesetting of web URLs.
%\usepackage{draftcopy} % Uncomment this line to have the
                        % word, "DRAFT," as a background
                        % "watermark" on all of the pages of
                        % of your draft versions. When ready
                        % to generate your final copy, re-comment
                        % it out with a percent sign to remove
                        % the word draft before you re-run
                        % Makediss for the last time.

\author{Anthony Michael Scopatz}    % Required

\address{2906 West Ave. \#16 \\ Austin, Texas 78705}  % Required

\title{Dominant Physics Fuel Cycle Modeling \& Analysis}    % Required


%%%%%%%%%%%%%%%%%%%%%%%%%%%%%%%%%%%%%%%%%%%%%%%%%%%%%%%%%%%%%%%%%%%%%%
% NOTICE: The total number of supervisors and other members %%%%%%%%%%
%%%%%%%%%%%%%%% MUST be seven (7) or less! If you put in more, %%%%%%%
%%%%%%%%%%%%%%% they are put on the page after the Committee %%%%%%%%%
%%%%%%%%%%%%%%% Certification of Approved Version page. %%%%%%%%%%%%%%
%%%%%%%%%%%%%%%%%%%%%%%%%%%%%%%%%%%%%%%%%%%%%%%%%%%%%%%%%%%%%%%%%%%%%%

%%%%%%%%%%%%%%%%%%%%%%%%%%%%%%%%%%%%%%%%%%%%%%%%%%%%%%%%%%%%%%%%%%%%%%
%
% Enter names of the supervisor and co-supervisor(s), if any,
% of your dissertation committee. Put one name per line with
% the name in square brackets. The name on the last line, however,
% must be in curly braces.
%
% If you have only one supervisor, the entry below will read:
%
%	\supervisor
%		{Supervisor's Name}
%
% NOTE: Maximum three supervisors. Minimum one supervisor.
% NOTE: The Office of Graduate Studies will accept only two supervisors!
% 
%
\supervisor
    {Erich Schneider}

%%%%%%%%%%%%%%%%%%%%%%%%%%%%%%%%%%%%%%%%%%%%%%%%%%%%%%%%%%%%%%%%%%%%%%
%
% Enter names of the other (non-supervisor) members(s) of your
% dissertation committee. Put one name per line with the name
% in square brackets. The name on the last line, however, must
% be in curly braces.
%
% NOTE: Maximum six other members. Minimum zero other members.
% NOTE: The Office of Graduate Studies may restrict you to a total
%	of six committee members.
%
%
\committeemembers
    [Steven Biegalski]
    [Sheldon Landsberger]
	[Mark Deinert]
	{Man-Sung Yim}

%%%%%%%%%%%%%%%%%%%%%%%%%%%%%%%%%%%%%%%%%%%%%%%%%%%%%%%%%%%%%%%%%%%%%%

\previousdegrees{M.S.E.}
     % The abbreviated form of your previous degree(s).
     % E.g., \previousdegrees{B.S., MBA}.
     %
     % The default value is `B.S., M.S.'

\graduationmonth{December}
     % Graduation month, either May, August, or December, in the form
     % as `\graduationmonth{May}'. Do not abbreviate.
     %
     % The default value (either May, August, or December) is guessed
     % according to the time of running LaTeX.

\graduationyear{2007}
     % Graduation year, in the form as `\graduationyear{2001}'.
     % Use a 4 digit (not a 2 digit) number.
     %
     % The default value is guessed according to the time of 
     % running LaTeX.

\typist{the author}
     % The name(s) of typist(s), put `the author' if you do it yourself.
     % E.g., `\typist{Maryann Hersey and the author}'.
     %
     % The default value is `the author'.


%%%%%%%%%%%%%%%%%%%%%%%%%%%%%%%%%%%%%%%%%%%%%%%%%%%%%%%%%%%%%%%%%%%%%%
% Commands for master's theses and reports.			     %
%%%%%%%%%%%%%%%%%%%%%%%%%%%%%%%%%%%%%%%%%%%%%%%%%%%%%%%%%%%%%%%%%%%%%%
%
% If the degree you're seeking is NOT Doctor of Philosophy, uncomment
% (remove the % in front of) the following two command lines (the ones
% that have the \ as their second character).
%
%\degree{MASTER OF ARTS}
%\degreeabbr{M.A.}

% Uncomment the line below that corresponds to the type of master's
% document you are writing.
%
%\masterreport
%\masterthesis


%%%%%%%%%%%%%%%%%%%%%%%%%%%%%%%%%%%%%%%%%%%%%%%%%%%%%%%%%%%%%%%%%%%%%%
% Some optional commands to change the document's defaults.	     %
%%%%%%%%%%%%%%%%%%%%%%%%%%%%%%%%%%%%%%%%%%%%%%%%%%%%%%%%%%%%%%%%%%%%%%
%
%\singlespacing
%\oneandonehalfspacing

%\singlespacequote
\oneandonehalfspacequote

\topmargin 0.125in  % Adjust this value if the PostScript file output
                    % of your dissertation has incorrect top and 
                    % bottom margins. Print a copy of at least one
                    % full page of your dissertation (not the first
                    % page of a chapter) and measure the top and
                    % bottom margins with a ruler. You must have
                    % a top margin of 1.5" and a bottom margin of
                    % at least 1.25". The page numbers must be at
                    % least 1.00" from the bottom of the page.
                    % If the margins are not correct, adjust this
                    % value accordingly and re-compile and print again.
                    %
                    % The default value is 0.125"

                    % If you want to adjust other margins, they are in the
                    % utdiss2-nn.sty file near the top. If you are using
                    % the shell script Makediss on a Unix/Linux system, make
                    % your changes in the utdiss2-nn.sty file instead of
                    % utdiss2.sty because Makediss will overwrite any changes
                    % made to utdiss2.sty.

%%%%%%%%%%%%%%%%%%%%%%%%%%%%%%%%%%%%%%%%%%%%%%%%%%%%%%%%%%%%%%%%%%%%%%
% Some optional commands to be tested.                               %
%%%%%%%%%%%%%%%%%%%%%%%%%%%%%%%%%%%%%%%%%%%%%%%%%%%%%%%%%%%%%%%%%%%%%%

% If there are 10 or more sections, 10 or more subsections for a section,
% etc., you need to make an adjustment to the Table of Contents with the
% command \longtocentry.
%
%\longtocentry 



%%%%%%%%%%%%%%%%%%%%%%%%%%%%%%%%%%%%%%%%%%%%%%%%%%%%%%%%%%%%%%%%%%%%%%
%	Some math support.                                               %
%%%%%%%%%%%%%%%%%%%%%%%%%%%%%%%%%%%%%%%%%%%%%%%%%%%%%%%%%%%%%%%%%%%%%%
%
%	Theorem environments (these need the amsthm package)
%
%% \theoremstyle{plain} %% This is the default

\newtheorem{thm}{Theorem}[section]
\newtheorem{cor}[thm]{Corollary}
\newtheorem{lem}[thm]{Lemma}
\newtheorem{prop}[thm]{Proposition}
\newtheorem{ax}{Axiom}

\theoremstyle{definition}
\newtheorem{defn}{Definition}[section]

\theoremstyle{remark}
\newtheorem{rem}{Remark}[section]
\newtheorem*{notation}{Notation}

%\numberwithin{equation}{section}


%%%%%%%%%%%%%%%%%%%%%%%%%%%%%%%%%%%%%%%%%%%%%%%%%%%%%%%%%%%%%%%%%%%%%%
%	Macros.							     %
%%%%%%%%%%%%%%%%%%%%%%%%%%%%%%%%%%%%%%%%%%%%%%%%%%%%%%%%%%%%%%%%%%%%%%
%
%	Here some macros that are needed in this document:


\newcommand{\latexe}{{\LaTeX\kern.125em2%
                      \lower.5ex\hbox{$\varepsilon$}}}

\newcommand{\amslatex}{\AmS-\LaTeX{}}

\chardef\bslash=`\\ % \bslash makes a backslash (in tt fonts)
                    %	p. 424, TeXbook

\newcommand{\cn}[1]{\texttt{\bslash #1}}

\makeatletter   % Starts section where @ is considered a letter
                % and thus may be used in commands.
\def\square{\RIfM@\bgroup\else$\bgroup\aftergroup$\fi
  \vcenter{\hrule\hbox{\vrule\@height.6em\kern.6em\vrule}%
                                              \hrule}\egroup}
\makeatother    % Ends sections where @ is considered a letter.
                % Now @ cannot be used in commands.

\makeindex      % Make the index

%%%%%%%%%%%%%%%%%%%%%%%%%%%%%%%%%%%%%%%%%%%%%%%%%%%%%%%%%%%%%%%%%%%%%%
%		The document starts here.                                    %
%%%%%%%%%%%%%%%%%%%%%%%%%%%%%%%%%%%%%%%%%%%%%%%%%%%%%%%%%%%%%%%%%%%%%%

\begin{document}

\copyrightpage  % Produces the copyright page.


%
% NOTE: In a doctoral dissertation, the Committee Certification page
%		(with signatures) is BEFORE the Title page.
%	In a masters thesis or report, the Signature page
%		(with signatures) is AFTER the Title page.
%
%	If you are writing a masters thesis or report, you MUST REVERSE
%	the order of the \commcertpage and \titlepage commands below.
%
\commcertpage   % Produces the Committee Certification
                % of Approved Version page (doctoral)
                % or Signature page (masters).
                %   20 Mar 2002 cwm

\titlepage      % Produces the title page.



%%%%%%%%%%%%%%%%%%%%%%%%%%%%%%%%%%%%%%%%%%%%%%%%%%%%%%%%%%%%%%%%%%%%%%
% Dedication and/or epigraph are optional, but must occur here.      %
%%%%%%%%%%%%%%%%%%%%%%%%%%%%%%%%%%%%%%%%%%%%%%%%%%%%%%%%%%%%%%%%%%%%%%
%
\begin{dedication}
\index{Dedication@\emph{Dedication}}%
For the Horde!
\end{dedication}


\begin{acknowledgments}		% Optional
\index{Acknowledgments@\emph{Acknowledgments}}%
Thank you.  I'll be here til Thursday.
\end{acknowledgments}


% The abstract is required. Note the use of ``utabstract'' instead of
% ``abstract''! This was necessary to fix a page numbering problem.
% The abstract heading is generated automatically.
% Do NOT use \begin{abstract} ... \end{abstract}.
%
\utabstract
\index{Abstract}%
\indent
This document has the form of a ``fake'' doctoral dissertation
in order to provide an example of such, but it is actually a
copy of Miguel Lerma's documentation for the Mathematics
Department Computer Seminar of 25 March 1998 updated in July 2001
and following by Craig McCluskey to meet the March 2001
requirements of the Graduate School.

This document and its source file show to write a Doctoral Dissertation using 
\LaTeX{} and the utdiss2 package. 



\tableofcontents   % Table of Contents will be automatically
                   % generated and placed here.

\listoftables      % List of Tables and List of Figures will be placed
\listoffigures     % here, if applicable.



%%%%%%%%%%%%%%%%%%%%%%%%%%%%%%%%%%%%%%%%%%%%%%%%%%%%%%%%%%%%%%%%%%%%%%
% Actual text starts here.					     %
%%%%%%%%%%%%%%%%%%%%%%%%%%%%%%%%%%%%%%%%%%%%%%%%%%%%%%%%%%%%%%%%%%%%%%
%
% Including external files for each chapter makes this document simpler,
% makes each chapter simpler, and allows for generating test documents
% with as few as zero chapters (by commenting out the include statements).
% This allows quicker processing by the Makediss command file in case you
% are not working on a specific, long and slow to compile chapter. You
% can even change the chapter order by merely interchanging the order
% of the include statements (something I found helpful in my own
% dissertation).
%
%\include{chapter-introduction}

%\include{chapter-instructions}

%\include{chapter-howtouse}

%\include{chapter-makingbib}

%\include{chapter-tables+figs}

%\include{chapter-math}


%%%%%%%%%%%%%%%%%%%%%%%%%%%%%%%%%%%%%%%%%%%%%%%%%%%%%%%%%%%%%%%%%%%%%%
% Appendix/Appendices                                                %
%%%%%%%%%%%%%%%%%%%%%%%%%%%%%%%%%%%%%%%%%%%%%%%%%%%%%%%%%%%%%%%%%%%%%%
%
% If you have only one appendix, use the command \appendix instead
% of \appendices.
%
\appendices
\index{Appendices@\emph{Appendices}}%

%\include{chapter-appendix1}

%\include{chapter-appendix2}

%\include{chapter-appendix3}


%%%%%%%%%%%%%%%%%%%%%%%%%%%%%%%%%%%%%%%%%%%%%%%%%%%%%%%%%%%%%%%%%%%%%%
% Generate the bibliography.                                         %
%%%%%%%%%%%%%%%%%%%%%%%%%%%%%%%%%%%%%%%%%%%%%%%%%%%%%%%%%%%%%%%%%%%%%%
%                                                                    %
% NOTE: For master's theses and reports, NOTHING is permitted to     %
%   come between the bibliography and the vita. The command          %
%   to generate the index (if used) MUST be moved to before          %
%   this section.                                                    %
%                                                                    %

%\nocite{*}
            % This command causes all items in the
            % bibliographic database to be added to
            % the bibliography, even if they are not
            % explicitly cited in the text.
            %
            % Not using because my bib file is generated
            % by Mendeley...which makes it huge.

\bibliographystyle{plain}   % Here the bibliography
\bibliography{library}      % is inserted.
\index{Bibliography@\emph{Bibliography}}%
%%%%%%%%%%%%%%%%%%%%%%%%%%%%%%%%%%%%%%%%%%%%%%%%%%%%%%%%%%%%%%%%%%%%%%


%%%%%%%%%%%%%%%%%%%%%%%%%%%%%%%%%%%%%%%%%%%%%%%%%%%%%%%%%%%%%%%%%%%%%%
% Generate the index.                                                %
%%%%%%%%%%%%%%%%%%%%%%%%%%%%%%%%%%%%%%%%%%%%%%%%%%%%%%%%%%%%%%%%%%%%%%
%                                                                    %
% NOTE: For master's theses and reports, NOTHING is permitted to     %
%   come between the bibliography and the vita. This section         %
%   to generate the index (if used) MUST be moved to before          %
%   the bibliography section.                                        %
%                                                                    %
\printindex % Include the index here. Comment out this line          %
%           % with a percent sign if you do not want an index.       %
%%%%%%%%%%%%%%%%%%%%%%%%%%%%%%%%%%%%%%%%%%%%%%%%%%%%%%%%%%%%%%%%%%%%%%


%%%%%%%%%%%%%%%%%%%%%%%%%%%%%%%%%%%%%%%%%%%%%%%%%%%%%%%%%%%%%%%%%%%%%%
% Vita page.							     %
%%%%%%%%%%%%%%%%%%%%%%%%%%%%%%%%%%%%%%%%%%%%%%%%%%%%%%%%%%%%%%%%%%%%%%

\begin{vita}
Victor Frankenstein is the son of Alphonse Frankenstein and Caroline Beaufort, who died of scarlet fever when Victor was young. Victor has two younger brothers \u2014 William, the youngest, who is killed by Victor's creation, and Ernest, the middle child, who wants to join the Foreign Service like a "true Genevese". Victor falls in love with Elizabeth, who became his adoptive "cousin."

As a young man, Frankenstein is interested in the works in alchemists such as Cornelius Agrippa, Paracelsus, and Albertus Magnus, and he longs to discover the fabled elixir of life. He loses interest in both these pursuits and in science as a whole after seeing the remains of a tree struck by lightning. However, at the University of Ingolstadt, Frankenstein develops a fondness for chemistry, and becomes obsessed with the idea of creating life in inanimate matter through artificial means, leaving university to pursue this goal for the next two years.

Assembling a humanoid creature perhaps by the use of a chemical, apparatus or a combination of both (he avoids the question three times when asked), Frankenstein successfully brings it to life only to be repulsed by its monstrous ugliness. He abandons and flees his creation, who disappears and soon embarks upon a journey of vengeance that results in the death of his younger brother, William. The creature approaches Frankenstein and begs him to create a female companion for him; Frankenstein agrees, but ultimately destroys this creation, aghast at the idea of a race of monsters. Enraged, the creature swears revenge and, on Frankenstein's wedding night, kills Elizabeth (his creator's new bride), as well as his best friend Henry Clerval.

Frankenstein pursues the "fiend" or "daemon" (as he calls his creation) to the Arctic with the intent of destroying it; he ultimately fails in his mission, however, when he falls through an ice flow and contracts severe pneumonia. He is rescued by a passing freighter, but dies after relating his tale to the ship's captain. His creature, upon discovering the death of its creator, is overcome by sorrow and vows to commit suicide by burning himself alive in "the Northernmost extremity of the globe"; he then disappears, never to be seen or heard from again.
\end{vita}

\end{document}
