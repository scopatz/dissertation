\chapter{Conclusion}
\index{Conclusion@\emph{Conclusion}}
\label{diss_conclusion}
Previous NFC simulators were either not physics-based models which could capture perturbations made to the system
or had prohibitively long run times when perturbations were introduced.  Because of these 
limitations, only static base cases and linear sensitivity analyses have historically been considered.  
Though such base cases were often chosen on the informed knowledge of the system designer, 
large gaps in the analysis space existed.  The essential physics models presented here address all such 
limitations.

Essential physics methods in the context of the nuclear fuel cycle are models 
in which only the internal parameters affecting system wide metrics (such as material flow
or electricity production) are explicitly solved for.  
Essential physics modeling approaches represent a balance of simplicity of fidelity as deemed 
appropriate based on the analysis needs of the modeler.
By restricting the subset of 
calculations performed, such methods are many orders of magnitude faster than first-principle 
methods.  Moreover, these models remain physically valid in the local range on which they are defined.
Thus perturbing their initial conditions yields appropriate responses to the final state
in ways that linear or statistical parametric fits are not capable of.

These essential physics reactor models have been used to examine three fuel cycles 
in depth: Once-Through, RU, and FR cases.  
Departing from discrete, pre-defined scenario studies, they were next used to perform 
a stochastic entropy-based sensitivity analysis on thirty design parameters in a fast
burner recycle scenario.  However, the one-energy group reactor model originally demonstrated, 
which was sufficient for fast reactors and (to a lesser degree) uranium-fueled light 
water reactors,  failed to capture 
spectral changes in the core as a function of burnup.
Therefore to enable the analysis of other fuel cycles, a multigroup reactor model was implemented.  

The analysis that was performed herein was facilitated by the essential physics models themselves.
Low-fidelity (lookup-table) models do not allow for perturbations that respond 
in a physics-based way.  High-fidelity models continue to have prohibitive execution times in the 
context of fuel cycle simulation.  By achieving fast, physics-based results these models increase 
the information generated by the corresponding simulations.
Moreover, the speed and fidelity at which these models operate \emph{enables} the 
new fuel cycle analytics used.  

In conclusion, these methods form a suite of modeling technologies which reach from the lowest levels
(individual components) to the highest (inter-cycle comparisons).
Prior to the development of this model suite, such broad-ranging analysis had been unrealistic to perform.
The work here thus presents a new, multi-scale approach to fuel cycle system design.  
Because of these essential physics models, it is now possible to tightly couple engineering 
concerns from multiple components simultaneously.  When assessing the impact of any 
proposed fuel cycle change, or when attempting to balance the relative merits of one 
cycle to another, the ability to track perturbation effects through the entire system becomes
essential.
