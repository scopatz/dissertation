\chapter{Conclusion}
\index{Conclusion@\emph{Conclusion}}%
\label{diss_conclusion}
Thus the essential physics reactor models have been used to examine three fuel cycles 
in depth: Once-Through, RU and FR cases.  This has in turn been utilized to perform 
a stochastic entropy-based sensitivity analysis on thirty design parameters in a fast
burner recycle scenario.  Moreover to enable the analysis other fuel cycle classes, 
a multigroup reactor model was implemented.  This final model demonstrated the capture 
of spectral changes in the core as a function of burnup.

The analysis that was performed herein was enabled by the essential physics models themselves.
Low-fidelity (lookup-table) models do not allow for perturbations that respond 
in a physics-based way.  High-fidelity models continue to have prohibitive execution times in the 
context of fuel cycle simulation.  By achieving fast, physics-based results these models increase 
the information generated by the corresponding simulations.
Moreover, the speed and fidelity at which these medium-fidelity models operate \emph{enables} the 
new fuel cycle analytics used.  

In conclusion, these methods form a stack of modeling technologies which reach from the lowest levels
(individual components) to the highest (inter-cycle comparisons).
Prior to the development of this model suite, such broad-ranging analysis has been unrealistic to perform.
